\documentclass{article}

\usepackage{amsmath}
\usepackage[parfill]{parskip} % this package kills the indentations

\usepackage[a4paper, textwidth=400pt, textheight=598pt]{geometry}

% \usepackage{setspace}
% \onehalfspacing

\title{Eigenvalues and Eigenvectors}
\author{huibosa}
\date{\today}

\begin{document}
\maketitle
\tableofcontents
\newpage

\section{Makov Matrices, Populations, and Economics}

This section is about \textbf{\textit{positive matrices}}: every $a_{ij}>0$, and \textbf{\textit{The largest eigenvalue is real and positive and so its eigenvector}}

\subsection{Makov Matrices}
The steady state equation $Au_{\infty}=u_{\infty}$ makes $u_{\infty}$ an eigenvector with eigenvalue 1:
\begin{flalign*}
	\quad \text{steady state}\quad \quad
	\begin{bmatrix}
		.8 & .3 \\
		.2 & .7
	\end{bmatrix}
	\begin{bmatrix}
		.6 \\
		.4
	\end{bmatrix}=
	\begin{bmatrix}
		.6 \\
		.4
	\end{bmatrix}=u_{\infty}
\end{flalign*}

\begin{align*}
  &&\boxed{
		\begin{array}{ll}
			\text{Makov Matirx} \quad \quad &
			\boxed{
				\parbox{20em}{
			Every entry of $A$ is positive: $a_{ij}>0$ \\
					Every column of $A$ adds to 1.
				}
			}
		\end{array}
	}
\end{align*}

Because of 1: Multiplying $u_0 > 0$ by A produces a nonnegative $u_1=Au_0\neq0$. \\
Because of 2: If the components of $u_0$ add to 1, so do the components of $u_1 = Au_0$.

\paragraph{Example 1}

The fraction of rental cars in Denver starts at $\frac{1}{50} = .02$. The fraction outside Denver is .98. Every month, 80\% of the Denver cars stay in Denver (and 20\% leave).  Also 5\% of the outside cars come in (95\% stay outside). This means that the fractions $u_0 = (.02, .98)$ are multiplied by A:

$u_{\infty}$ is an eigenvector of $A$ corresponding to $\lambda=1$

\paragraph{Solution}

The eigenvalues are $\lambda_1=1$ and $\lambda_2=.75$

\begin{flalign*}
	\parbox{10em}{$Ax=\lambda x$}
	A
	\begin{bmatrix}
		.2 \\
		.8
	\end{bmatrix}=
	1
	\begin{bmatrix}
		.2 \\
		.8
	\end{bmatrix}
	\quad \quad \text{and} \quad \quad
	A
	\begin{bmatrix}
		-1 \\
		1
	\end{bmatrix}=
	.75
	\begin{bmatrix}
		-1 \\
		1
	\end{bmatrix}
\end{flalign*}


\end{document}
