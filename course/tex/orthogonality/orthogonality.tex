\documentclass{article}

\usepackage{amsmath}
\usepackage[parfill]{parskip} % this package kills the indentations

\usepackage[a4paper, textwidth=400pt, textheight=598pt]{geometry}

% \usepackage{setspace}
% \onehalfspacing

\title{Othogonality}
\author{huibosa}
\date{\today}

\begin{document}
\maketitle
\tableofcontents
\newpage

\section{Orthogonality of Four Subspaces}

Two vectors are orthognal when their dot product is zero:
\[\parbox{12em}{Orthogonal vectors}
  v^Tw=0 \qquad and \qquad
  \|v\|^{2}+\|w\|^{2}=\|v+w\|^{2}
\]
The right side is $(v+w)^{T}(v+w)$, which equals $v^w+w^T$ when $v^tw=w^tv=0$.

A matrix multiplies a vector: $A$ times $x$. At the first level this is only numbers. At
the second level $Ax$ is a combination of column vectors. The third level shows subspaces.

\textbf{The row space is perpendicular to nullspaces.} Every row of $A$ is perpendicular to every solution of $Ax$.

\textbf{The column space is perpendicular to the nullspace of $A$.} When $b$ is outside the column space--when we want to solve $Ax=b$ and can't do it--then this nullspace of $AT$ comes into its own. It contains the error $e=b-Ax$ in the "least-squares" solution.

\paragraph{DEFINITION}

Two subspaces $V$ and $W$ of a vector space are orthogonal if every vector $v$ in $V$ is perpendicular to ever y vector $w$ in $W$:
\[\parbox{12em}{Orthogonal subspaces}
  v^tw=0 \quad \text{for all $v$ in $V$ and all $w$ in $W$}
\]

\paragraph{Example 1}

The floor of your room (extended to infinity) is a subspace $V$. The line where two walls meet is a subspace $W$ (one-dimensional). Those subspaces are orthogonal. Every vector up the meeting line of the walls is perpendicular to every vector in the floor.

\paragraph{Example 2}

Two walls look perpendicular but those two subspaces are not orthogonal! The meeting line is in both $V$ and $W$. and this line is not perpendicular to itself. \textit{Two planes (dimensions 2 and 2 in $R^3)$ cannot be orthogonal subspaces}.

When a vectors is in two orthogonal subspaces it \textit{must} be zero.

\newpage

\section{Projections}

\fbox{\parbox{\textwidth}{
		\begin{enumerate}
      \item The projection of $a$ vector $b$ onto the line through $a$ is the closest point $p=a(a^Tb/a^Ta)$.
      \item The error $e=b-p$ is perpendicular to $a$ : Right triangle $b$ $p$ $e$ has $\|p\|^{2}+\|e\|^{2}=\|b\|^{2}$
      \item The \textbf{projection} of $b$ onto $a$ subspace $S$ is the closest vector $p$ in $S;$ $b-p$ is orthogonal to $S$.
      \item $A^TA$ is invertible (and symmetric) only if $A$ has independent columns : $N(A^TA)=N(A)$
      \item Then the projection of $b$ onto the column space of $A$ is the vector $p=A(A^TA)^{-1}A^Tb$.
      \item The \textbf{projection matrix} onto $C(A)$ is $P=A(A^TA)^{-1}AT.$ It has $p=Pb$ and $p^2=p=p^T$
		\end{enumerate}
}
}

\fbox{\parbox{8em}{Projection matrix onto z axis:}}
\[ 
P_1=
\begin{bmatrix}
  0 & 0 & 0 \\
  0 & 0 & 0 \\
  0 & 0 & 1
\end{bmatrix}
\quad \quad
\textit{Onto the $xy$ plane:} \quad \quad
P_2=
\begin{bmatrix}
  1 & 0 & 0 \\
  0 & 1 & 0 \\
  0 & 0 & 0
\end{bmatrix}
\]

\end{document}
