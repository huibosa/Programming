\documentclass{article}

\usepackage{amsmath}
\usepackage[parfill]{parskip} % this package kills the indentations

\usepackage[a4paper, textwidth=400pt, textheight=598pt]{geometry}
% \geometry{a4paper,textwidth=345pt,textheight=598pt}

\usepackage{setspace}
\onehalfspacing

\title{Eigenvalues and Eigenvectors}
\author{huibosa}
\date{\today}

\begin{document}
\maketitle
\tableofcontents
\newpage

\section{The Properties of Determinants}

For a 2 by 2 matrix:

\[
	A=
	\begin{bmatrix}
		a & b \\
		c & d
	\end{bmatrix}
	\quad \text{has inverse} \quad
	A^{-1}=\frac{1}{ad-bc}
	\begin{bmatrix}
		d  & -b \\
		-c & a
	\end{bmatrix}
\]
The product of the pivots is the \textit{determinant}:
\begin{flalign*}
	\parbox[c]{15em}{\textbf{Product of pivots}}
	a\left( d-\frac{e}{a}b=ad-bc \right)
	\quad \textbf{which is}
	\quad \text{det}(A). &  &
\end{flalign*}

\subsection{The Properties of the Determinants}

\paragraph{1}
\textbf{\textit{determinant of the $n$ by $n$ identity matrix is $1$}}.

\paragraph{2}
\textbf{\textit{The determinant changes sign when two rows are exchanged}} (sign reversal).

\paragraph{3}
\textbf{\textit{The determinant is a linear function of each row separately}} (all other rows stay fixed).
\begin{flalign*}
	                & \parbox[c]{20em}{\textbf{Multiply row 1 by any number $t$   \\ det is multiplied by $t$}}
	\begin{vmatrix}
		ta & tb \\
		c  & d
	\end{vmatrix}
	=t
	\begin{vmatrix}
		a & b \\
		c & d
	\end{vmatrix}                                                                \\
	                & \parbox[c]{20em}{\textbf{Add row 1 of $A$ to row 1 of $A'$: \\ then determinants add}}
	\begin{vmatrix}
		a+a' & b+b' \\
		c    & d
	\end{vmatrix}
	=
	\begin{vmatrix}
		a & b \\
		c & d
	\end{vmatrix}
	+
	\begin{vmatrix}
		a' & b' \\
		c  & d
	\end{vmatrix} &
\end{flalign*}

This rule does not mean $\text{det}(2I)=2det(I)$:

\[
	\begin{vmatrix}
		2 & 0 \\
		0 & 2
	\end{vmatrix}=2^2=4
	\quad \text{and} \quad
	\begin{vmatrix}
		t & 0 \\
		0 & t
	\end{vmatrix}=t^2
\]

This is like area and volume. Expand a rectangle by 2 and its area increases by 4. Expand an n-dimensional box by $t$ and its volume increases by $t^{n}$

\paragraph{4}
\textbf{\textit{If two rows of $A$ are equal, then $\text{det}(A)=0$}} (all other rows stay fixed).

\paragraph{5}
\textbf{\textit{Substracting a multiple of one row from another row leaves $det(A)$ unchanged}}.

\paragraph{Conclusion}
\textit{The determinant is not changed by the usual elimination steps from $A$ to $U$. Thus $\text{det}A$ equals $\text{det}U$.} Every row exchange reverse the sign, so always $\text{det}(A)=\pm\ \text{det}(U)$.

\paragraph{6}
\textbf{\textit{A matrix with rows of zeroes have $\text{det}(A)=0$}}.

\paragraph{7}
\textbf{\textit{If A is triangular then $\text{det}(A)=a_{11}a_{22}\cdots a_{nn}=product\ of\ diagonal\ entries$}}.

\paragraph{8}
\textbf{\textit{If $A$ is singular then $\text{det}(A)=0$. If $A$ is invertible then $\text{det}(A)\neq0$}}.

The product of nonzero pivots gives a nonzero determinants:
\begin{flalign*}
	\fbox{
		\parbox[c]{10em}{\textbf{Multiply pivots}}
		\fbox{
			$\text{det}(A)=\pm\ \text{det}(U)=\pm (product\ of\ the pivots)$
	}} &  &
\end{flalign*}

The pivots of a 2 by 2 matrix (if $a\neq 0$)are $a$ and $d-(c/a)b$:

\[
	\parbox{13em}{\textbf{The determinant is}}
	\begin{vmatrix}
		a & b \\
		c & d
	\end{vmatrix}
	=\begin{vmatrix}
		a & b        \\
		0 & d-(c/a)b
	\end{vmatrix}
	=ad-bc
\]

The sign in $\pm\ \text{det}(U)$ depends on whether the number of row exchange is even or odd: $+1$ or $-1$ is the determinant of the permutation $P$ that exchange rows.

With no row exchanges, $P=I$ and $\text{det}(A)=\text{det}(U)=product\ of\ pivots$. And $\text{det}(L)=1$:
\[
	\text{If}\quad
	PA=LU
	\quad\text{then}\quad
	\text{det}(P)\text{det}(A)=\text{det}(L)\text{det}(U)
	\quad\text{and}\quad
	\text{det}(A) = \pm\ \text{det}(U)
\]

\paragraph{9}
\textbf{\textit{The determinant of $AB$ is $\text{det}(A)$ times $\text{det}(B)$}}: $|AB|=|A||B|$
\begin{flalign*}
	\parbox[c]{10em}{\textbf{ Product rule}}
	\begin{vmatrix}
		a & b \\
		c & d
	\end{vmatrix}
	\begin{vmatrix}
		p & q \\
		r & s
	\end{vmatrix}
	=
	\begin{vmatrix}
		ap + br & aq + bs \\
		cp + dr & eq + ds
	\end{vmatrix} &  &
\end{flalign*}

The determinant of $A^{-1}$ is $1/\text{det}(A)$.

\paragraph{10}
\textbf{\textit{The transpose of $A$ has the same determinant of $A$.}}

\paragraph{Important comment on columns}
\fbox{Every rule for the rows can apply to the columns.}

\section{Permutations and Cofactors}


\end{document}
